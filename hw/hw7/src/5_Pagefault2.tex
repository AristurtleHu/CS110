
The system implements a single-level page table for 32-bit virtual addresses in byte-addressable memory, using 4KB pages with 32-bit page table entries (PTEs) that reserve 8 bits for protection and valid flags. Given the program segment:
{\bf   \begin{lstlisting}[label={lst:4_C_src}]
int w[24][64];
...
for ( i=0; i<24; i++ )
    for ( j=0; j<64; j++ ) w[i][j]=10;

    \end{lstlisting}}
Assume the array w[24][64] (row-major order, starting at $\mathtt{0x392E00}$, 4-byte integers) is initially not present in main memory, with no page replacement during execution.
\begin{parts}
    \part[12] After execution: [12 points]
    \begin{itemize}
        \vspace{0.5cm}
        \item The array spans \blank[3cm]{2} pages in the virtual address space.
              \vspace{0.5cm}
        \item The number of page faults triggered is \blank[3cm]{2}.
              \vspace{0.5cm}
        \item Fault addresses (page base address): \blank[7cm]{0x392E00, 0x3D2E00}.
    \end{itemize}

    \part[8] If the page size is 128 bytes, and the array is stored in column-major order, after execution: [8 points]
    \begin{itemize}
        \vspace{0.5cm}
        \item The number of page faults triggered is \blank[3cm]{64}.
              \vspace{0.5cm}
    \end{itemize}
    \vspace{0.5cm}

    \part[8] Assume page size is 16KB with two-level page table. What is the maximum size of the physical memory that can be supported by this computer? [8 points]

    The offset is 14 bits, and the page table entry (PTE) is 32 bits. The maximum number of pages is $2^{32-14} = 2^{18}$, and each PTE can map to a physical frame of size $2^{14}$ bytes. Therefore, the maximum physical memory size is: $$2^{18} \times 2^{14} = 2^{32} \text{ bytes} = 4 \text{ GB}.$$

\end{parts}
